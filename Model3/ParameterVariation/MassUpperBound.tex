\subsection{Considering effect on model is mass is $m=125000kg$}

\subsubsection{Solutions of the DE}
\textbf{Case1: }
Using the equation from 4.2 for case 1, we see that:
\begin{align*}
    v &= \frac{m}{c_{1}m+kt}
\end{align*}

Then using the point (t=0, v=96), we see that:
\begin{center}
\begin{align*}
    v &= \frac{m}{kt+cm}
    \\ 96 &= \frac{1}{c_1}
    \\ c &= \frac{1}{96}
\end{align*}
\end{center}
Using the point (t=9, v=55), we see that:
\begin{center}
\begin{align*}
    v &= \frac{m}{kt+c_1m}
    \\ 55 &= \frac{m}{9k + c_1m} = \frac{96m}{864k + m} && \text{As $c=\frac{1}{96}$}
    \\ 864k &= \frac{96}{55}m - m
    \\ k &= \frac{1}{864}(\frac{96}{55}m - m)
    \\ &= \text{107.84 $N/ms^{-1}$}
\end{align*}
\end{center}
From this we can deduce that for case 1, if $m=125000kg$:
\begin{center}
\begin{align*}
    v &= \frac{m}{kt+c_{1}m}
    \\ v &= \frac{1}{\frac{k}{m}t+\frac{1}{96}}
    \\ \Aboxed{v &= \frac{96}{0.0828t + 1}}
\end{align*}
\end{center}

\textbf{Case 2: } 
Using the equation from 4.2 for case 2, we see that:

\begin{align*}
    \sqrt{\frac{k}{F_b}} \arctan{(v\sqrt{\frac{k}{F_b}})} &= -\frac{k}{m}t + c_2
\end{align*}
Then using the point (t=26, v=0), we see that:
\begin{center}
\begin{align*}
    \sqrt{\frac{k}{F_b}} \arctan{(v\sqrt{\frac{k}{F_b}})} &= -\frac{k}{m}t + c_2
    \\ -\frac{26k}{m} + c_2 &= 0
    \\ c_2 &= \frac{26k}{m} 
    \\ &= 0.02243
\end{align*}
\end{center}
This means our equation now becomes:
\begin{equation}
    \sqrt{\frac{k}{F_b}} \arctan{(v\sqrt{\frac{k}{F_b}})} = -\frac{k}{m}t + 0.02243
\end{equation}
\\ \\
After this using the point (t=9, v=55), we see that:
\\
\begin{center}
\begin{align*}
    \sqrt{\frac{k}{F_b}} \arctan{(v\sqrt{\frac{k}{F_b}})} &= -\frac{k}{m}t + 0.02243
    \\ \sqrt{\frac{k}{F_b}} \arctan{(55\sqrt{\frac{k}{F_b}})} &= -9\frac{k}{m} + 0.02243
\end{align*}
\end{center}
To find $F_b$, we can use Newton-Raphson, by letting $\alpha=\sqrt{\frac{k}{F_b}}$ and substituting $9 \frac{k}{m} = 7.765\times{10^{-3}}$
\begin{align*}
    \alpha \arctan{(55\alpha)} &= -9 \frac{k}{m} + 0.02243
\end{align*}
\begin{equation}
    \alpha \arctan{(55\alpha)} - 0.014664 = 0
\end{equation}
Thus by applying Newton-raphson on the above equation, we will be able to find alpha, giving us a value for $F_b$

\subsubsection{Using Newton-Raphson to find $\alpha$}
\begin{align*}
    \text{let } f(\alpha) &= \alpha \arctan{(55\alpha)} - 0.014664
    \\ \text{Therefore } f'(\alpha) &= \arctan{55a} + \frac{55\alpha}{1+3025\alpha^2}
\end{align*}

\begin{figure}[H]
\centering

\begin{tikzpicture}
\begin{axis}[
    %axis lines = left,
    xlabel = x,
    ylabel = $f(x)$,
]
%Below the red parabola is defined
\addplot [
    domain=0:0.02,
    samples=100, 
    color=red,
]{x*rad(atan(55*x))-0.01465};
\addlegendentry{$xatan{55x}-0.01545$}
\end{axis}
\end{tikzpicture}
\caption{Graph showing the graph of $f(x)$}
\end{figure}

Using the graph of $f(x)$, we see that the root is close to 0.02. Therefore we will use 0.02 as our initial guess for the root.
\\ \\
Using the iterative formula, $x_{i+1} := x_{i}-\frac{f(x)}{f'(x)}$, to find the next closest root we can produce a table showing all the iterations and the next approximation of the root it gave.

\begin{table}[h]
\centering
    \begin{tabular}{|l|l|l|l|l|}
        \hline
        i & $x_i$ & $f(x)$ & $f'(x)$ & $x_{i+1}$ \\ \cline{1-5}
        1 & 0.02000 & 0.00200 & 1.33072 & 0.01850 \\ \cline{1-5}
        2 & 0.01850 & 0.00003 & 1.29401 & 0.01848 \\ \cline{1-5}
        3 & 0.01848 & 0.00000 & 1.29346 & 0.01848 \\ \cline{1-5}
    \end{tabular}
    \caption{Newton Raphson Iterations}
\end{table}
\
Thus we find that alpha is 0.01848, meaning that the fraction \\ $\sqrt{\frac{k}{F_b}} = 0.01848$ to 5dp.
\newline \newline
Therefore by rearranging for v, we can conclude that for case 2, when $m=125000kg$:
\begin{align*}
    \alpha \arctan{(\alpha v)} &= -\frac{k}{m}t + 0.02243
    \\ \arctan{(\alpha v)} &= -\frac{k}{m\alpha}t + \frac{0.02233}{\alpha}
    \\ v &= \frac{1}{\alpha} \tan{(-\frac{k}{m\alpha}t + \frac{0.02243}{\alpha})}
    \\ \Aboxed{v &=  54.1\tan{(-0.0466t + 1.21)}}
\end{align*}
