\subsection{Comparison}
Clearly, it can be seen that using the assumption that $R \propto v$ does produces a model that gives a much more accurate graph that fits the data. Furthermore, the RMS error has also decreased from 3.74 to 1.78, showing that the new model is a much more improved model compared to our initial model.

For case1, where $0 \leq t \leq 9$, it can be seen that the graph produced from the model, does however still has a lot residual, which can be easily observed. The residual are also above 1 showing that the predictions are not close to the true values and can be improved but also for some points the residuals are extremely huge. An example of this is when t=4. When t=4, we have that the residual is 2.95, which is very close to 3. This overestimation in case1 shows that this model can be improved again by relaxing the assumptions further.

Unfortunately, for case2 however, when the brakes were applied the predictions were much less than they were supposed to be. The model underestimated most of the values which again further shows that although the model has improved it can be improved further by using a better model, which could be achieved by relaxing assumptions further.